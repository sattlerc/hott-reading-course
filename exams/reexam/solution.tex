\documentclass[11pt]{article}

\usepackage[left=3cm,right=3cm,top=3cm,bottom=3cm]{geometry}
\usepackage{amssymb,amsmath,amsthm}
\usepackage{mathtools}
\usepackage{cleveref}
\usepackage{enumitem}
\usepackage{tabularx}
\usepackage{tikz-cd}
\usepackage{parskip}

\newcolumntype{Y}{>{\centering\arraybackslash}X}

\DeclarePairedDelimiter\trunc\lVert\rVert

\newcommand{\U}{\mathcal{U}}
\newcommand{\N}{\mathbb{N}}
\newcommand{\Z}{\mathbb{Z}}
\newcommand{\F}{\mathbb{F}}
\newcommand{\id}{\mathsf{id}}
\newcommand{\ap}{\mathsf{ap}}

\newcommand{\judgeq}{\mathrel{\dot{=}}}

\newcommand{\Fin}{\mathsf{Fin}}
\newcommand{\refl}{\mathsf{refl}}
\newcommand{\base}{\mathsf{base}}
\newcommand{\concat}{\mathsf{concat}}
\newcommand{\Sloop}{\mathsf{loop}}
\newcommand{\indeq}{\mathsf{ind\mbox{-}eq}}
\newcommand{\inv}{\mathsf{inv}}

\newcommand{\inl}{\mathsf{inl}}
\newcommand{\inr}{\mathsf{inr}}

\newcommand{\Unit}{\mathsf{1}}
\newcommand{\unit}{\star}

\newcommand{\Nsucc}{S}
\newcommand{\Nadd}{\mathsf{add}}
\newcommand{\Nmul}{\mathsf{mul}}

\newcommand{\fst}{\mathsf{pr}_1}
\newcommand{\snd}{\mathsf{pr}_2}

\newcommand{\isEquiv}{\mathsf{is\mbox{-}equiv}}
\newcommand{\isConst}{\mathsf{is\mbox{-}constant}}

\newcommand{\fix}{\mathsf{fix}}

\newlist{conditions}{enumerate}{1}
\setlist[conditions]{label=(\arabic*),itemsep=0ex}
\Crefname{conditionsi}{Condition}{Conditions}
\crefname{conditionsi}{condition}{conditions}

\newlist{problems}{enumerate}{1}
\setlist[problems]{label=\arabic*.,ref=\arabic*}
\Crefname{problemsi}{Problem}{Problems}
\crefname{problemsi}{Problem}{Problems}

\pagenumbering{gobble}

\begin{document}

\title{Introduction to homotopy type theory: re-exam}
\author{DAT235/DIT577/PhD reading course}
\date{2024, March 13}

\maketitle

\begin{itemize}
\item
Grade scale:\qquad
\begin{tabularx}{10cm}{|c|Y|Y|Y|Y|}
  \hline
  Fraction of points & $\geq 0$ & $\geq 2/5$ & $\geq 3/5$ & $\geq 4/5$
  \\\hline
  Grade & U & 3 & 4 & 5
  \\\hline
\end{tabularx}
\item
Time: 4 hours
\item
No aids allowed.
\item
You may use familiar facts from the course book or our discussions without justification, provided they do not already include the statement to be proven or depend on it.
\item
The axioms of function extensionality and univalence may only be used where stated. 
\end{itemize}

\newpage

\begin{problems}

\item \label{path-associative}
\textbf{[4 points]}
Consider a type $A$.
The equality type of $A$ has an induction principle involving
\[
\indeq_{a,P} : P(a, \refl_a) \to \prod_{x : A} \prod_{p : a = x} P(x, p)
\]
for $a : A$ and a family of types $P(x, p)$ indexed by $x : A$ and $p : a = x$.

Define composition of identifications in $A$ (you can choose its with judgmental behaviour).
Explicitly state the parameter $P$ when you use $\indeq$.

{\color{purple}
For example, we define
\begin{align*}
\concat &: \prod_{x, y, z : A} (x = y) \to (y = z) \to (x = z)
\\
\concat(x, y, z, p, q) &= J_{P,x}(y, \lambda z', q'.\, q', y, p, z, q)
\end{align*}
where
\[
P(y', p') = \prod_{z' : A} \prod_{q' : y' = z'} (y' = z') \to (x = z')
.\]
This uses the book convention of applying an iterated function to a tuple of arguments.
}

\item \label{function-extensionality}
\textbf{[4 points]}
Consider a type $A$ and a family $B$ of types over $A$.
State the axiom of extensionality for dependent functions from $a : A$ to $B(a)$.
You may use the notion of equivalence without explanation, but everything else needs to be defined.

{\color{purple}
Let $f$ and $g$ be dependent functions from $a : A$ to $B(a)$.
We have a function
\[
h_{f, g} : f = g \to \prod_{a : A} f(a) = g(a)
\]
defined by identification induction via
\[
h_{f, f}(\refl_f)(a) = \refl_{f(a)}
.\]
The axiom of function extensionality states that $h_{f, g}$ is an equivalence.
}

\item \label{path-split-equivalence}
\textbf{[4 points]}
Let $f : A \to B$ be a map such that:
\begin{conditions}
\item \label{path-split-equivalence:section} $f$ has a section,
\item \label{path-split-equivalence:path-section} for $x, y : A$, the map $\ap_f : (x =_A y) \to (f(x) =_B f(y))$ has a section.
\end{conditions}
Prove that $f$ is an equivalence (bi-invertible).

{\color{purple}
By \cref{path-split-equivalence:section}, $f$ has a section $s : B \to A$.
We show that $s$ is also a retraction of $f$.
This means $s(f(a)) = a$ for $a : A$.
Using \cref{path-split-equivalence:path-section}, it suffices to show $f(s(f(a))) = f(a)$.
But this holds since $f(s(b)) = b$ for $b : B$.
}

\item \label{set-coproduct}
\textbf{[4 points]}
Consider sets $A$ and $B$.
Show that the coproduct $A + B$ is again a set.
You may use the characterization of identifications in $A + B$ from the course.

{\color{purple}
Given $x_0, x_1 : A + B$, we have to show that $x_0 = x_1$ is a proposition.
By equality induction it suffices to show that $x = x$ is a proposition for $x : A + B$.
We use coproduct elimination on $x$.

\begin{itemize}
\item
For $x \judgeq \inl(a)$ with $a : A$, we have that $x = x$ is equivalent to $a = a$.
This is a proposition since $A$ is a set.
\item
For $x \judgeq \inr(b)$ with $b : B$, we have that $x = x$ is equivalent to $b = b$.
This is a proposition since $B$ is a set.
\end{itemize}

We finish by recalling that being a proposition is invariant under equivalence.
}

\item \label{yoneda}
\textbf{[4 points]}
Consider a type $A$ and a univalent universe $\U$ containing the identity types of $A$.
Consider the function $v : A \to \U^A$ sending $x$ to $\lambda y.\,y =_A x$.
Show that the action of $v$ on identifications has a section.
You may use function extensionality.

{\color{purple}
Given $x_0, x_1 : A$, we must show that the action
\[
\ap_v : x_0 = x_1 \longrightarrow v(x_0) =_{\U^A} v(x_1)
\]
of $v$ on identifications has a retraction.
We will more generally show that it is invertible.

By 2-out-of-3, it suffices to show that the composition with the function extensionality equivalence
\[
h_{v(x_0), v(x_1)} : v(x_0) =_{\U^A} v(x_1) \longrightarrow \prod_{y : A} (y = x_0) =_U (y = x_1)
\]
from Problem~\ref{function-extensionality} is invertible.
We further compose with the equivalence
\[
(y = x_0) =_U (y = x_1) \longrightarrow (y = x_0) \simeq (y = x_1)
\]
from univalence.
It remains to show that the map 
\[
x_0 = x_1 \longrightarrow \prod_{y : A} (y = x_0) \simeq (y = x_1)
\]
sending reflexivity to the family of identity equivalences is invertible.
By 2-out-of-3, it sufices to show that the map
\[
\prod_{y : A} (y = x_0) \simeq (y = x_1) \longrightarrow x_0 = x_1
\]
evaluating at $x_0$ and reflexivity is invertible.

Every family of maps $e_y : (y = x_0) \to (y = x_1)$ for $y : A$ is a family of equivalences.
To see this, we check that the induced map
\[
\sum_{y:A} (y = x_0) \to \sum_{y:A} (y = x_1)
\]
on total spaces is invertible.
But both sides here are contractible singletons.

It remains to show that the map
\[
\prod_{y : A} ((y = x_0) \to (y = x_1)) \longrightarrow x_0 = x_1
\]
evaluating at $x_0$ and reflexivity is invertible.
This is equivalently
\[
\prod_{(y, p) : \sum_{y : A} y = x_0} y = x_1 \longrightarrow x_0 = x_1
\]
evaluating at $(x_0, \refl)$.
This is invertible because $\sum_{y : A} y = x_0$ is a contractible singleton with center $(x_0, \refl)$.
}

\item \label{finite-types-add-point}
\textbf{[4 points]}
Let $\F$ be the univalent universe of finite types.
Construct an equivalence
\[
\F \simeq \sum_{X:\F} X
.\]
You may use function extensionality.

{\color{purple}
In the forward direction, we define $f : \F \to \sum_{x : \F} X$ by sending $A$ to $(\Unit + A, \inl(\unit))$.
In the reverse direction, we define $g : \sum_{x : \F} X \to \F$ by sending $X : \F$ with $x_0 : X$ to the type $\sum_{x : X} x \neq x_0$.
Recall that finite types are sets with decidable equality.
Therefore, $g(X, x_0)$ is a decidable subtype of $X$, which is again finite.

We now check that these maps are inverse to each other.
Given $A : \F$, using univalence of $\F$, we need to show that $\sum_{x : \Unit + A} x \neq \inl(\unit)$ is equivalent to $A$.
Distributing the sum over the coproduct and using the characterization of identifications in coproducts, this is the coproduct of $\sum_{z : \Unit} z \neq \unit$ and $\sum_{a : A} \neg \emptyset$.
Simplifying further, this is coproduct of the empty type and $A$, or just $A$.

Given $X : \F$ with $x_0 : X$, using univalence and the characterization of identifications in dependent sums, we need to construct an equivalence $e : \unit + \sum_{x : X} x \neq x_0 \simeq X$ such that $e(\inl(\unit)) = x_0$.
This suggests defining $e(\inl(\unit)) \judgeq x_0$ and $e(\inr(x, p)) = x$.

}

\end{problems}

\end{document}
