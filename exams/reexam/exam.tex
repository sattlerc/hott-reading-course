\documentclass[11pt]{article}

\usepackage[left=3cm,right=3cm,top=3cm,bottom=3cm]{geometry}
\usepackage{amssymb,amsmath,amsthm}
\usepackage{mathtools}
\usepackage{cleveref}
\usepackage{enumitem}
\usepackage{tabularx}
\usepackage{tikz-cd}
\usepackage{parskip}

\newcolumntype{Y}{>{\centering\arraybackslash}X}

\DeclarePairedDelimiter\trunc\lVert\rVert

\newcommand{\U}{\mathcal{U}}
\newcommand{\N}{\mathbb{N}}
\newcommand{\Z}{\mathbb{Z}}
\newcommand{\F}{\mathbb{F}}
\newcommand{\id}{\mathsf{id}}
\newcommand{\ap}{\mathsf{ap}}

\newcommand{\Fin}{\mathsf{Fin}}
\newcommand{\refl}{\mathsf{refl}}
\newcommand{\base}{\mathsf{base}}
\newcommand{\Sloop}{\mathsf{loop}}
\newcommand{\indeq}{\mathsf{ind\mbox{-}eq}}
\newcommand{\inv}{\mathsf{inv}}

\newcommand{\Nsucc}{S}
\newcommand{\Nadd}{\mathsf{add}}
\newcommand{\Nmul}{\mathsf{mul}}

\newcommand{\fst}{\mathsf{pr}_1}
\newcommand{\snd}{\mathsf{pr}_2}

\newcommand{\isEquiv}{\mathsf{is\mbox{-}equiv}}
\newcommand{\isConst}{\mathsf{is\mbox{-}constant}}

\newcommand{\fix}{\mathsf{fix}}

\newlist{conditions}{enumerate}{1}
\setlist[conditions]{label=(\arabic*),itemsep=0ex}
\Crefname{conditionsi}{Condition}{Conditions}
\crefname{conditionsi}{condition}{conditions}

\newlist{problems}{enumerate}{1}
\setlist[problems]{label=\arabic*.,ref=\arabic*}
\Crefname{problemsi}{Problem}{Problems}
\crefname{problemsi}{Problem}{Problems}

\pagenumbering{gobble}

\begin{document}

\title{Introduction to homotopy type theory: re-exam}
\author{DAT235/DIT577/PhD reading course}
\date{2024, March 13}

\maketitle

\begin{itemize}
\item
Grade scale:\qquad
\begin{tabularx}{10cm}{|c|Y|Y|Y|Y|}
  \hline
  Fraction of points & $\geq 0$ & $\geq 2/5$ & $\geq 3/5$ & $\geq 4/5$
  \\\hline
  Grade & U & 3 & 4 & 5
  \\\hline
\end{tabularx}
\item
Time: 4 hours
\item
No aids allowed.
\item
You may use familiar facts from the course book or our discussions without justification, provided they do not already include the statement to be proven or depend on it.
\item
The axioms of function extensionality and univalence may only be used where stated. 
\end{itemize}

\newpage

\begin{enumerate}

\item \label{path-associative}
\textbf{[4 points]}
Consider a type $A$.
The equality type of $A$ has an induction principle involving
\[
\indeq_{a,P} : P(a, \refl_a) \to \prod_{x : A} \prod_{p : a = x} P(x, p)
\]
for $a : A$ and a family of types $P(x, p)$ indexed by $x : A$ and $p : a = x$.

Define composition of identifications in $A$ (you can choose its with judgmental behaviour).
Explicitly state the parameter $P$ when you use $\indeq$.

\item \label{function-extensionality}
\textbf{[4 points]}
Consider a type $A$ and a family $B$ of types over $A$.
State the axiom of extensionality for dependent functions from $a : A$ to $B(a)$.
You may use the notion of equivalence without explanation, but everything else needs to be defined.

\item \label{path-split-equivalence}
\textbf{[4 points]}
Let $f : A \to B$ be a map such that:
\begin{conditions}
\item \label{path-split-equivalence:section} $f$ has a section,
\item \label{path-split-equivalence:path-section} for $x, y : A$, the map $\ap_f : (x =_A y) \to (f(x) =_B f(y))$ has a section.
\end{conditions}
Prove that $f$ is an equivalence (bi-invertible).

\item \label{set-coproduct}
\textbf{[4 points]}
Consider sets $A$ and $B$.
Show that the coproduct $A + B$ is again a set.
You may use the characterization of identifications in $A + B$ from the course.

\item \label{yoneda}
\textbf{[4 points]}
Consider a type $A$ and a univalent universe $\U$ containing the identity types of $A$.
Consider the function $v : A \to \U^A$ sending $x$ to $\lambda y.\,y =_A x$.
Show that the action of $v$ on identifications has a section.
You may use function extensionality.

\item \label{finite-types-add-point}
\textbf{[4 points]}
Let $\F$ be the univalent universe of finite types.
Construct an equivalence
\[
\F \simeq \sum_{X:\F} X
.\]
You may use function extensionality.

\end{enumerate}

\end{document}
